%%%%%%%%%%%%%%%%%%%%%%%%%%%%%%%%%%%%%%%%%
% Lachaise Assignment
% LaTeX Template
% Version 1.0 (26/6/2018)
%
% This template originates from:
% http://www.LaTeXTemplates.com
%
% Authors:
% Marion Lachaise & François Févotte
% Vel (vel@LaTeXTemplates.com)
%
% License:
% CC BY-NC-SA 3.0 (http://creativecommons.org/licenses/by-nc-sa/3.0/)
% 
%%%%%%%%%%%%%%%%%%%%%%%%%%%%%%%%%%%%%%%%%

%----------------------------------------------------------------------------------------
%	PACKAGES AND OTHER DOCUMENT CONFIGURATIONS
%----------------------------------------------------------------------------------------

\documentclass{article}

%%%%%%%%%%%%%%%%%%%%%%%%%%%%%%%%%%%%%%%%%
% Lachaise Assignment
% Structure Specification File
% Version 1.0 (26/6/2018)
%
% This template originates from:
% http://www.LaTeXTemplates.com
%
% Authors:
% Marion Lachaise & François Févotte
% Vel (vel@LaTeXTemplates.com)
%
% License:
% CC BY-NC-SA 3.0 (http://creativecommons.org/licenses/by-nc-sa/3.0/)
% 
%%%%%%%%%%%%%%%%%%%%%%%%%%%%%%%%%%%%%%%%%

%----------------------------------------------------------------------------------------
%	PACKAGES AND OTHER DOCUMENT CONFIGURATIONS
%----------------------------------------------------------------------------------------

\usepackage{amsmath,amsfonts,stmaryrd,amssymb} % Math packages

\usepackage{enumerate} % Custom item numbers for enumerations

\usepackage[ruled]{algorithm2e} % Algorithms

\usepackage[framemethod=tikz]{mdframed} % Allows defining custom boxed/framed environments

\usepackage{listings} % File listings, with syntax highlighting
\lstset{
	basicstyle=\ttfamily, % Typeset listings in monospace font
}

%----------------------------------------------------------------------------------------
%	DOCUMENT MARGINS
%----------------------------------------------------------------------------------------

\usepackage{geometry} % Required for adjusting page dimensions and margins

\geometry{
	paper=a4paper, % Paper size, change to letterpaper for US letter size
	top=2.5cm, % Top margin
	bottom=3cm, % Bottom margin
	left=2.5cm, % Left margin
	right=2.5cm, % Right margin
	headheight=14pt, % Header height
	footskip=1.5cm, % Space from the bottom margin to the baseline of the footer
	headsep=1.2cm, % Space from the top margin to the baseline of the header
	%showframe, % Uncomment to show how the type block is set on the page
}

%----------------------------------------------------------------------------------------
%	FONTS
%----------------------------------------------------------------------------------------

\usepackage[utf8]{inputenc} % Required for inputting international characters
\usepackage[T1]{fontenc} % Output font encoding for international characters

\usepackage{XCharter} % Use the XCharter fonts

%----------------------------------------------------------------------------------------
%	COMMAND LINE ENVIRONMENT
%----------------------------------------------------------------------------------------

% Usage:
% \begin{commandline}
%	\begin{verbatim}
%		$ ls
%		
%		Applications	Desktop	...
%	\end{verbatim}
% \end{commandline}

\mdfdefinestyle{commandline}{
	leftmargin=10pt,
	rightmargin=10pt,
	innerleftmargin=15pt,
	middlelinecolor=black!50!white,
	middlelinewidth=2pt,
	frametitlerule=false,
	backgroundcolor=black!5!white,
	frametitle={Command Line},
	frametitlefont={\normalfont\sffamily\color{white}\hspace{-1em}},
	frametitlebackgroundcolor=black!50!white,
	nobreak,
}

% Define a custom environment for command-line snapshots
\newenvironment{commandline}{
	\medskip
	\begin{mdframed}[style=commandline]
}{
	\end{mdframed}
	\medskip
}

%----------------------------------------------------------------------------------------
%	FILE CONTENTS ENVIRONMENT
%----------------------------------------------------------------------------------------

% Usage:
% \begin{file}[optional filename, defaults to "File"]
%	File contents, for example, with a listings environment
% \end{file}

\mdfdefinestyle{file}{
	innertopmargin=1.6\baselineskip,
	innerbottommargin=0.8\baselineskip,
	topline=false, bottomline=false,
	leftline=false, rightline=false,
	leftmargin=2cm,
	rightmargin=2cm,
	singleextra={%
		\draw[fill=black!10!white](P)++(0,-1.2em)rectangle(P-|O);
		\node[anchor=north west]
		at(P-|O){\ttfamily\mdfilename};
		%
		\def\l{3em}
		\draw(O-|P)++(-\l,0)--++(\l,\l)--(P)--(P-|O)--(O)--cycle;
		\draw(O-|P)++(-\l,0)--++(0,\l)--++(\l,0);
	},
	nobreak,
}

% Define a custom environment for file contents
\newenvironment{file}[1][File]{ % Set the default filename to "File"
	\medskip
	\newcommand{\mdfilename}{#1}
	\begin{mdframed}[style=file]
}{
	\end{mdframed}
	\medskip
}

%----------------------------------------------------------------------------------------
%	NUMBERED QUESTIONS ENVIRONMENT
%----------------------------------------------------------------------------------------

% Usage:
% \begin{question}[optional title]
%	Question contents
% \end{question}

\mdfdefinestyle{question}{
	innertopmargin=1.2\baselineskip,
	innerbottommargin=0.8\baselineskip,
	roundcorner=5pt,
	nobreak,
	singleextra={%
		\draw(P-|O)node[xshift=1em,anchor=west,fill=white,draw,rounded corners=5pt]{%
		Question \theQuestion\questionTitle};
	},
}

\newcounter{Question} % Stores the current question number that gets iterated with each new question

% Define a custom environment for numbered questions
\newenvironment{question}[1][\unskip]{
	\bigskip
	\stepcounter{Question}
	\newcommand{\questionTitle}{~#1}
	\begin{mdframed}[style=question]
}{
	\end{mdframed}
	\medskip
}

%----------------------------------------------------------------------------------------
%	WARNING TEXT ENVIRONMENT
%----------------------------------------------------------------------------------------

% Usage:
% \begin{warn}[optional title, defaults to "Warning:"]
%	Contents
% \end{warn}

\mdfdefinestyle{warning}{
	topline=false, bottomline=false,
	leftline=false, rightline=false,
	nobreak,
	singleextra={%
		\draw(P-|O)++(-0.5em,0)node(tmp1){};
		\draw(P-|O)++(0.5em,0)node(tmp2){};
		\fill[black,rotate around={45:(P-|O)}](tmp1)rectangle(tmp2);
		\node at(P-|O){\color{white}\scriptsize\bf !};
		\draw[very thick](P-|O)++(0,-1em)--(O);%--(O-|P);
	}
}

% Define a custom environment for warning text
\newenvironment{warn}[1][Warning:]{ % Set the default warning to "Warning:"
	\medskip
	\begin{mdframed}[style=warning]
		\noindent{\textbf{#1}}
}{
	\end{mdframed}
}

%----------------------------------------------------------------------------------------
%	INFORMATION ENVIRONMENT
%----------------------------------------------------------------------------------------

% Usage:
% \begin{info}[optional title, defaults to "Info:"]
% 	contents
% 	\end{info}

\mdfdefinestyle{info}{%
	topline=false, bottomline=false,
	leftline=false, rightline=false,
	nobreak,
	singleextra={%
		\fill[black](P-|O)circle[radius=0.4em];
		\node at(P-|O){\color{white}\scriptsize\bf i};
		\draw[very thick](P-|O)++(0,-0.8em)--(O);%--(O-|P);
	}
}

% Define a custom environment for information
\newenvironment{info}[1][Info:]{ % Set the default title to "Info:"
	\medskip
	\begin{mdframed}[style=info]
		\noindent{\textbf{#1}}
}{
	\end{mdframed}
}
 % Include the file specifying the document structure and custom commands

\usepackage{wrapfig, booktabs}
\usepackage{array}
\usepackage{pgfplots, pgfplotstable}
\usepackage{floatrow}
\usepackage[makeroom]{cancel}


%----------------------------------------------------------------------------------------
%	ASSIGNMENT INFORMATION
%----------------------------------------------------------------------------------------

\title{PHYS207: Simple Harmonic Motion} % Title of the assignment

\author{Andrei Tumbar \\
Robert-Jason Pearsall \\
Cyrus Uedoi} % Author name and email address

\date{}

%----------------------------------------------------------------------------------------

\begin{document}

\maketitle

%----------------------------------------------------------------------------------------
%	INTRODUCTION
%----------------------------------------------------------------------------------------

\section*{Abstract} % Unnumbered section

The purpose of this lab was to utilize two different methods of determining the spring constant of a provided spring. The first method, ‘Static’, consists of recording how different hanging masses affected the length of the spring. The second method, ‘Dynamic’, involved recording how hanging differing values of masses on the spring affected the spring-mass system’s simple harmonic oscillations.

\section{Static Measurements}

\begin{wraptable}{r}{5.5cm}
\caption{Data collected for static experiment.}
\begin{tabular}{c|c|c}
Mass (g) & Length (cm) & $\Delta x$ (cm) \\\hline
50 & 21.6 & 4.6 \\
100 & 27.7 & 10.7 \\
250 & 45.5 & 28.5 \\
150 & 33.5 & 16.5 \\
175 & 36.4 & 19.4 \\
200 & 39.4 & 22.4 \\
300 & 51.4 & 34.4 \\
\end{tabular}
\end{wraptable}

Because a spring is defined with the following force equation: $F_s(x) = -kx$. The slope of the Mass vs Displacement graph will yield the $k$ of the spring.

\[
1.19 \frac{\cancel{kg}}{m} \cdot \frac{9.81 N}{\cancel{kg}} = \boxed{11.674 \frac{N}{m}} 
\]

\pgfplotstableread{
X Y
.050 .046
.100 .107
.250 .285
.150 .165
.175 .194
.200 .224
.300 .344
}\datatable

Figure 1 shows the graph generated by ploting mass vs displacement. As expected a straight line is generated indicating that the $F_s(x) = kx$ is a good model for this spring. \\\\\\

\begin{figure}[h]
\caption{Mass vs Displacement}
\begin{tikzpicture}
\begin{axis}[
    xlabel={Mass on the spring (kg)},
    ylabel={Displacement (m)},
    xmin=0, xmax=.350,
    ymin=0, ymax=.40,
    xtick={0,.050,.100,.150,.200,.250,.300},
    ytick={0,.10,.20,.30,.40},
    legend pos=north west,
    ymajorgrids=true,
    grid style=dashed,
]

\addplot[only marks, mark = *] table {\datatable};
\addplot [thick] table[
    y={create col/linear regression={y=Y}}
] {\datatable};
\addlegendentry{$data$}
\addlegendentry{%
$\pgfmathprintnumber[precision=3]{\pgfplotstableregressiona} \cdot x
\pgfmathprintnumber[print sign]{\pgfplotstableregressionb}$}

\end{axis}
\end{tikzpicture}
\end{figure}



\section{Dynamic Measurements}

\begin{wraptable}{r}{5.5cm}
\caption{Data collected for dynamic experiment.}
\begin{tabular}{c|c|r}
Mass (g) & $T^2$ ($s^2$) & Error (\%) \\\hline
550 & 2.98 & $\pm$ 9.85 \\
250 & 1.45 & $\pm$ 14.93 \\
300 & 1.74 & $\pm$ 2.28 \\
100 & 0.74 & $\pm$ 13.90 \\
150 & 0.98 & $\pm$ 0.000 \\
200 & 1.24 & $\pm$ 10.78 \\
50 & 0.48 & $\pm$ 21.57 \\
275 & 1.63 & $\pm$ 2.35 \\

\end{tabular}
\end{wraptable}

To derive the period of a spring as it oscillates we need to start at the definition of a spring.



\begin{gather}
F_s(x) = -kx \\
m\frac{d^2x}{dt^2} = -kx \\ 
\frac{d^2x}{dt^2} = \frac{-kx}{m}
\end{gather}



If an object is experiencing a force in the form below, we know that it must be in harmonic oscillation. With a frequency of $\omega$.

\[
\frac{d^2x}{dt^2} = -\omega^2x
\]

\pgfplotstableread{
X Y E
.250 1.45 .1493
.300 1.74 .0228
.100 0.74 .1390
.150 0.98 0
.200 1.24 .1078
.050 0.48 .2157
.275 1.63 .0235
}\datatable



\begin{wrapfigure}{R}{0.5\textwidth}
\caption{Mass vs Period$^2$}
\begin{tikzpicture}
\begin{axis}[
    xlabel={Mass on the spring (kg)},
    ylabel={Period$^2$ ($s^2$)},
    xmin=0, xmax=.350,
    ymin=0, ymax=3,
    xtick={0,.050,.100,.150,.200,.250,.300},
    ytick={0,0.5,1,1.5,2,2.5,3},
    legend pos=north west,
    ymajorgrids=true,
    grid style=dashed,
]

\addplot[only marks, mark = *] plot [error bars/.cd, y dir = both, y explicit] table[x=X, y=Y, y error=E] {\datatable};
\addplot [thick] table[
    y={create col/linear regression={y=Y}}
] {\datatable};
\addlegendentry{$data$}
\addlegendentry{%
$\pgfmathprintnumber[precision=3]{\pgfplotstableregressiona} \cdot x
\pgfmathprintnumber[print sign]{\pgfplotstableregressionb}$}

\end{axis}
\end{tikzpicture}
\end{wrapfigure}

Because are differential equation is of the same form we can say:

\begin{gather}
\omega = \sqrt{\frac{k}{m}} \\
T = \frac{2\pi}{\omega} \\
T = 2\pi\sqrt{\frac{m}{k}}
\end{gather}

To linearize this function, we graph T$^2$ instead of T on the y-axis. Figure 2 is a graph of the following function:

\[
T^2(m) = \frac{4\pi^2}{k}(\frac{m}{1000})
\]

Mass is divided by 1000 to convert it to kilograms. The slope of the regression in figure 2 will be equivalent to $\frac{4\pi^2}{k}$.

\begin{gather}
5.012 \cdot 10^{-3} = \frac{4\pi^2}{1000k} \\
k = \frac{4\pi^2}{5.012} = 7.876 \frac{N}{m}
\end{gather}
\linebreak
\linebreak
\linebreak
\linebreak
\section*{Conclusion}


\end{document}
